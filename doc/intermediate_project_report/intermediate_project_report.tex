\documentclass[11pt]{article}
\usepackage{acl2014}
\usepackage{times}
\usepackage{url}
\usepackage{latexsym}

% Change this if needed to set titlebox size.
%\setlength\titlebox{5cm}

\title{LING 575: Intermediate Project Report}

\author{Claire Jaja \\
  University of Washington \\
  Seattle, WA \\
  {\tt cjaja@uw.edu} \\\And
  Andrea Kahn \\
  University of Washington \\
  Seattle, WA \\
  {\tt amkahn@uw.edu} \\}

\date{}

\begin{document}
\maketitle
\begin{abstract}
\end{abstract}

\section{Introduction}

Sentiment analysis techniques are typically developed on English text.  However, there is a proliferation of text in other languages as well, which could benefit from sentiment analysis.  Current approaches to sentiment analysis in languages other than English often involve automatically translating the text into English, which is likely to introduce errors, as well as increasing runtimes.  Some techniques rely on manually developed resources which are only available in English; however, others rest on machine learning algorithms that can easily be applied to other languages.  In this paper, we propose an approach to evaluate the effectiveness of these techniques when transferred directly to other languages.

\section{Approach}

\section{Results}

This section should present the major results of the formal evaluation of your system and components.

\section{Discussion}


\section{Conclusion}

\nocite{*}
\bibliographystyle{acl}
\bibliography{references}

%\begin{thebibliography}{}

%\end{thebibliography}

\end{document}
